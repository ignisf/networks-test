\begin{questions}
  \question Напишете най-характерното за работата на един комутатор
  (\foreignlanguage{english}{switch}).

  %\makeemptybox{2cm}

  \question Напишете най-характерното за работата на един маршрутизатор
  (\foreignlanguage{english}{router}).
  %\makeemptybox{2cm}

  \question Напишете най-малко три разлики между комутатор и маршрутизатор.
  %\makeemptybox{3cm}

  \question Опишете поведението на маршрутизатор в режим на динамична размяна на
  маршрути.
  %\makeemptybox{3cm}

  \question Опишете IP адресите на интерфейсите и съдържанието на маршрутните
  таблици на трите маршрутизатора, ако те са конфигурирани така, че да има
  двупосочна комуникация между PC 1 и PC 2.

  \begin{center}
    \begin{tikzpicture}[
      align=center,node distance=2cm,
      start chain=going right,
      diagram item/.style={
        on chain,
        join
      }
      ]

      \node [
      diagram item,
      label=above:Router 1,label=267:eth0,label=-15:eth1
      ] {\includegraphics{router}};

      \node [
      start branch=1 going below left,
      diagram item,
      label=below:PC 1,label=30:eth0
      ] {\includegraphics{pc}};

      \node [
      diagram item,
      label=above:Router 2,label=-165:eth1,label=-15:eth0
      ] {\includegraphics{router}};

      \node [
      diagram item,
      label=above:Router 3,label=-165:eth0,label=273:eth1
      ] {\includegraphics{router}};

      \node [
      start branch=1 going below right,
      diagram item,
      label=below:PC 2,label=150:eth0
      ] {\includegraphics{pc}};

    \end{tikzpicture}
  \end{center}
  %\makeemptybox{5cm}

  \question Дайте примери за IGP и EGP протоколи за маршрутизация. Напишете
  кратко пояснение за всеки пример.
  %\makeemptybox{6cm}

  \question Какво наричаме мрежова маска?
  %\makeemptybox{1cm}

  \question Дефинирайте три вида NAT и дайте пример за приложението на всеки от
  тях.
  %\makeemptybox{5cm}

  \question Напишете броя валидни IP адреси за следните IPv4 мрежови маски:
  \begin{itemize}
    \item \texttt{/25}
    \item \texttt{/20}
    \item \texttt{/19}
    \item \texttt{/17}
  \end{itemize}

  \question Разделете мрежата \texttt{10.2.4.0/22} на 4 подмрежи и напишете
  адресите и маските на тези мрежи.
  %\makeemptybox{1cm}

  \question Имаме мрежа с 500 хоста. Изберете мрежова маска, която осигурява
  най-малкото адресно пространство, което да ги побере.
  %\makeemptybox{1cm}

  \question Обяснете в какво се състои тристранното ръкостискане
  (\foreignlanguage{english}{three-way handshake}) при TCP.

  \question Имате Linux-базирана система с един мрежови интерфейс, на който е
  зададен IP адресът \texttt{192.168.0.2/255.255.255.240}. До какво ще доведе
  изпълнението на командата \texttt{route add -net 10.0.0.0 netmask 255.240.0.0
    gw 192.168.0.16}?

  \question Ако имаме мрежата \texttt{192.168.1.0/24}, на колко подмрежи ще бъде
  разделена тя, с помощта на маска с дължина \texttt{/26}?

  \question Посочете адресите за разпръскване
  (\foreignlanguage{english}{broadcast addresses}) на всяка подмрежа от
  предишния въпрос.

  \question Каква е целта на полето \texttt{time-to-live} в хедъра на IP?

  \question Каква е функционалността на маршрута по премълчаване
  (\foreignlanguage{english}{default route})?

  \question Вярно ли е, че тъй като TCP не може да разчита на IP да форматира
  пакетите правилно, IP адресът на дестинацията трябва да бъде включен в TCP
  хедъра?

  \question Какво наричаме автономна система
  (\foreignlanguage{english}{autonomous system, AS})?

  \question Какво разпръсква в мрежата маршрутизатор, който е конфигуриран да
  разчита на \foreignlanguage{english}{link-state} протокол за маршрутизация? А
  ако е конфигуриран с \foreignlanguage{english}{distance-vector} протокол за
  маршрутизация?
\end{questions}

%%% Local Variables:
%%% mode: latex
%%% TeX-master: "test"
%%% ispell-dictionary: "bulgarian"
%%% TeX-PDF-mode: t
%%% End:
