\documentclass[11pt,addpoints]{exam}

\usepackage[utf8]{inputenc}
\usepackage[T2A,T1]{fontenc}
\usepackage{amsmath}
\usepackage[english,bulgarian]{babel}

\makeatletter

\newcounter{points}
\def\vsolword#1{\def\@vsolword{#1}}

\def\@ansloop{
  \stepcounter{question}
  \ref{question@\arabic{question}} & & & \pointsofquestion{\arabic{question}} \\
  \hline
  \ifnum \value{question} < \numquestions\relax
    \@ansloop
  \fi
}

\def\soltable{
  \begin{tabular}{|c|c||c|c|}
    \hline
    {\@vqword} & {\@vsolword} & {\@vsword} & {\@vpword} \\
    \hline
    \setcounter{points}{0}
    \setcounter{question}{0}
    \@ansloop

    \multicolumn{1}{c|}{} & {\@vtword} & & {\numpoints} \\ \cline{2-4}
  \end{tabular}%
}

\makeatother

\vqword{Въпрос №}
\vsolword{Отговор(и)}
\vsword{Резултат}
\vpword{Макс. точки}
\vtword{Общо:}

\begin{document}

\pagestyle{headandfoot}
\firstpageheader
  {\large\bfseries Компютърни мрежи\\15 юли 2013}
  {}
  {Факултетен №:\enspace\makebox[2cm]{\dotfill},
    поток:\enspace\makebox[1cm]{\dotfill},
    група:\enspace\makebox[1cm]{\dotfill}\\
    Име:\enspace\makebox[8.8cm]{\dotfill}}

\firstpagefooter{}{}{}

\section*{Инструкции}

\begin{itemize}
\item Разполагате с 90 минути за работа.
\item Попълнете верните отговори на въпросите в колоната "`\textbf{Отговор(и)}"'
  и оставете колоната "`Резултат"' празна. Ще бъде преглеждана \emph{единствено}
  таблицата за оценяване.
\item Всеки нечетим или двусмислен отговор се счита за грешен.
\item Корекциите на отговори са позволени, стига да се спазва горното правило.
\end{itemize}

\section*{Таблица за оценяване}

\begin{center}
  \soltable
\end{center}

\cleardoublepage

\section*{Въпроси}

\begin{questions}

  % input questions here

\end{questions}

\end{document}
